\begin{czech}
\begin{abstract}

Párování prohledává kombinace nabídky a poptávky, které řadí dle míry, s jakou nabídka vyhovuje poptávce.
Tato práce demonstruje, jak lze dva obecné postupy, jmenovitě případové usuzování a statistické relační učení, použít pro párování veřejných zakázek a uchazečů o zakázky.
V obou případech párování využívá jak logické, tak statistické usuzování využívající vzájemně porovnatelná, polo-strukturovaná a sémanticky popsaná data.
Na základech případového usuzování jsme navrhli novou metodu párování implementovanou pomocí jazyka SPARQL pro dotazování v datech ve formátu RDF.
Metoda využívá podobnostní vyhledávání učící se z dříve udělených zakázek.
Pro párování vycházející ze statistického relačního učení jsme převzali algoritmus RESCAL pro faktorizaci multi-relačních tenzorů využívající kolektivní učení pro predikci vazeb.
Náš přínos v obou přístupech zahrnuje zejména výběr a tvorbu příznaků a také ladění parametrů párování.

Metody párování jsme aplikovali na soubor propojených otevřených dat veřejné správy, jehož ústředním prvkem je Věstník veřejných zakázek České republiky.
Doménu veřejných zakázek jsme zvolili, protože poskytuje explicitně popsané poptávky, které jsou díky zákonem požadovanému proaktivnímu zveřejňování oznámení o veřejných zakázkách dostupné v podobě otevřených a strukturovaných dat.
Náš výzkum je motivován rozsáhlým pasivním plýtváním ve veřejných zakázkách, které má párování šanci zmírnit návrhy efektivnější alokace veřejných prostředků.
Příprava použitých dat si však vyžádala rozsáhlé úsilí při tvorbě komplexních ETL procesů.
Jako rámec integrace těchto dat jsme využili propojená otevřená data.
Klíčové problémy dat jsme řešili pomocí technik pro propojování a fúzi dat.
Otestovali jsme a integrovali dostupný software založený na technologiích sémantického webu, ale také vyvinuli přepoužitelné nástroje pro předzpracování dat ve formátu RDF.

Evaluaci přesnosti a diverzity metod párování jsme provedli na úloze predikce vítězných uchazečů o zakázky v retrospektivních datech o zakázkách udělených během deseti let.
Kvalita a rozsah vstupních dat se projevily jako zásadní faktory rozhodující o úspěšnosti párování.
Párování využívající SPARQL ve všech ohledech jednoznačně překonalo přístup založený na algoritmu RESCAL, a to zejména s ohledem na diverzitu výsledků a náročnost výpočtu.
Na rozdíl od většiny využitých příznaků, které se projevily jako šum, se příznaky z řízených slovníků popisujících zakázky nebo uchazeče ukázaly pro párování jako podstatně informativnější.
Na hodnotu propojených dat poukázaly nejlepší výsledky u obou přístupů, které byly dosaženy párováním kombinujícím příznaky z více datových zdrojů.

\emph{Klíčová slova:} párování, propojená data, otevřená data, veřejné zakázky

\end{abstract}
\end{czech}
