\begin{czech}
\begin{abstract}

Párování prohledává možné páry nabídky a poptávky a řadí je dle míry, s jakou nabídka vyhovuje poptávce.
Práce demonstruje, jak lze dva obecné postupy, jmenovitě případové usuzování a statistické relační učení, použít pro párování veřejných zakázek a uchazečů o zakázky.
Oba osvojené postupy využívají jak logické, tak statistické usuzování pro párování vzájemně porovnatelných, polo-strukturovaných a sémanticky popsaných dat.
Navrhli jsme novou metodu párování založenou na případové usuzování, která je implementována pomocí dotazovacího jazyka SPARQL pro data ve formátu RDF.
Metoda využívá podobnostní vyhledávání učící se z v minulosti udělených zakázek, které jsou brány jako zkušenosti vyřešených problémů.
Pro párování vycházejícího ze statistického relačního učení jsme přejali RESCAL, což je algoritmus pro faktorizaci multi-relačních tenzorů, který využívá kolektivní učení pro predikci vazeb.
Náš přínos v obou přístupech zahrnuje zejména výběr a tvorbu příznaků a také ladění konfigurací pro párování.

Metody párování jsme aplikovali na soubor propojených otevřených dat veřejné správy, jehož ústředním prvkem je Věstník veřejných zakázek.
Doménu veřejných zakázek jsme si zvolili, protože poskytuje explicitně popsané poptávky, které jsou díky zákony vyžadovanému proaktivnímu zveřejňování oznámení o veřejných zakázkách dostupné v podobě otevřených a strukturovaných dat.
Náš výzkum je motivován rozsáhlým pasivním plýtváním ve veřejných zakázkách, které má párování šanci zmírnit návrhy efektivnější alokace veřejných prostředků.
Věstník veřejných zakázek jsme integrovali s dalšími daty veřejné správy, jako jsou číselníky nebo rejstříky právních osob.
Příprava dat si vyžádala rozsáhlé úsilí na budování komplextní ETL procesů, jednak z důvodu mnoha problémů kvality dat o veřejných zakázkách, ale také kvůli nesourodosti kombinovaných datových sad.
Jako rámec datové integrace jsme využili propojená otevřená data, která staví na technologických standardech sémantického webu.
Řešení klíčových problémů dat zahrnovalo návrh a implementaci technik pro propojování a fúzi dat.
V průběhu přípravy dat jsme otestovali a integrovali dostupný software založený na technologiích sémantického webu, ale také vyvinuli přepoužitelné nástroje pro předzpracování dat ve formátu RDF.

Evaluaci metod párování jsme provedli na predikci vítězných uchazečů o zakázky v retrospektivních datech o zakázkách udělených během doby 10 let.
Evaluací metrik přesnosti a diverzity jsme vyhodnotili přínos dílčích faktorů týkajících se metod párování, jako je například expanze dotazů nebo objem dat pro strojové učení.
Kvalita a rozsah vstupních dat se projevily jako zásadní faktory rozhodující o úspěšnosti párování.
Přístup k párování využívající jazyk SPARQL ve všech ohledech jednoznačně překonal přístup založený na algoritmu RESCAL, a to zejména s ohledem na diverzitu výsledků a výpočetní náročnost.
Na rozdíl od většiny využitých příznaků, které se projevily jako šum, se příznaky z řízených slovníků popisujících zakázky nebo uchazeče ukázaly jako podstatně informativnější pro párování.
Na hodnotu propojených dat poukázaly nejlepší výsledky u obou přístupů, které byly dosaženy párováním kombinujícím příznaky z více datových zdrojů.

\emph{Klíčová slova:} párování, propojená data, otevřená data, veřejné zakázky

\end{abstract}
\end{czech}
